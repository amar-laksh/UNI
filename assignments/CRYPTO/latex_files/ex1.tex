\documentclass[10pt,a4paper,oneside]{article}
\usepackage[utf8]{inputenc}
\usepackage{amsmath}
\usepackage{amsfonts}
\usepackage{amssymb}
\usepackage{makeidx}
\usepackage{graphicx}
\usepackage{lmodern}
\usepackage{listings}
\usepackage{color}
\usepackage{hyperref}

\lstset{language=Python}
\lstset{frame=lines}
\lstset{label={lst:code_direct}}
\lstset{basicstyle=\footnotesize}

\author{Amar Lakshya}
\title{Cryptography module, Exercises 1 (unassessed) - Answers}
\begin{document}

\part*{Cryptography module, Exercises 1 (unassessed) - Answers}

\textbf{All the code contained in this exercise solution is hosted at:\linebreak
\href{https://github.com/amar-laksh/UNI/tree/master/assignments/CRYPTO/code}
{https://github.com/amar-laksh/UNI/assignments/CRYPTO/code}}
\section*{Answer - 1:}

\textbf{Listing 1} contains python code to brute-force the given cipher-text "\textit{AVUEVLETSEISBNACBOOLEOBTILBDLCOBOOE}".

\lstset{caption={ex1.1.py}}
\begin{lstlisting}
from math import ceil

def encode(msg, key):
    cipher = ""
    for rails in range(0, key):
        for char in range(rails, len(msg), key):
            cipher += msg[char]
    return cipher

def decode(cipher, key):
    msg = ""
    count = ceil(len(cipher)/key)
    for rails in range(0, count):
        for c in range(rails, len(cipher), count):
            msg += cipher[c]
    return msg

def crack_cipher(cipher):
    for key in range(1, len(cipher)):
        print(decode(cipher, key))

cipher = "AVUEVLETSEISBNACBOOLEOBTILBDLCOBOOE"

crack_cipher(cipher)
\end{lstlisting}
\textbf{Listing 2} contains the output of the python code in \textbf{Listing 1} containing the message, "\textit{ALICELOVESBOBBUTBOBDOESNOTLOVEALICE}" shown in the bold.


\lstset{caption={output of ex1.1.py}}
\lstset{keywordstyle=\bfseries}
\lstset{morekeywords={ALICELOVESBOBBUTBOBDOESNOTLOVEALICE}}
\begin{lstlisting}
AVUEVLETSEISBNACBOOLEOBTILBDLCOBOOE
AOVLUEEOVBLTEITLSBEDILSCBONBAOCOBEO
ABIVNLUABECDVBLLOCEOOTLBSEOEOOIBEST
AEODVILLUSECEBOOVNBBLATOECIOTBLESOB
ATAOLVSCBCUEBTOEIOIBVSOLOLBLBOENEDE
AEBOIOVTNLLBUSAEBOEECODOVIBBLELSOTC
ALICELOVESBOBBUTBOBDOESNOTLOVEALICE
ALICELOVESBOBBUTBOBDOESNOTLOVEALICE
AVSBBEILOVLENOOLCOUEIAOBBOEETSCLTDB
AVSBBEILOVLENOOLCOUEIAOBBOEETSCLTDB
AVSBBEILOVLENOOLCOUEIAOBBOEETSCLTDB
AEEEBCOOIDOOVVTINBLBLLBEULSSAOETBCO
AEEEBCOOIDOOVVTINBLBLLBEULSSAOETBCO
AEEEBCOOIDOOVVTINBLBLLBEULSSAOETBCO
AEEEBCOOIDOOVVTINBLBLLBEULSSAOETBCO
AEEEBCOOIDOOVVTINBLBLLBEULSSAOETBCO
AEEEBCOOIDOOVVTINBLBLLBEULSSAOETBCO
AUVESIBABOEBIBLOOEVELTESNCOLOTLDCBO
AUVESIBABOEBIBLOOEVELTESNCOLOTLDCBO
AUVESIBABOEBIBLOOEVELTESNCOLOTLDCBO
AUVESIBABOEBIBLOOEVELTESNCOLOTLDCBO
AUVESIBABOEBIBLOOEVELTESNCOLOTLDCBO
AUVESIBABOEBIBLOOEVELTESNCOLOTLDCBO
AUVESIBABOEBIBLOOEVELTESNCOLOTLDCBO
AUVESIBABOEBIBLOOEVELTESNCOLOTLDCBO
AUVESIBABOEBIBLOOEVELTESNCOLOTLDCBO
AUVESIBABOEBIBLOOEVELTESNCOLOTLDCBO
AUVESIBABOEBIBLOOEVELTESNCOLOTLDCBO
AUVESIBABOEBIBLOOEVELTESNCOLOTLDCBO
AUVESIBABOEBIBLOOEVELTESNCOLOTLDCBO
AUVESIBABOEBIBLOOEVELTESNCOLOTLDCBO
AUVESIBABOEBIBLOOEVELTESNCOLOTLDCBO
AUVESIBABOEBIBLOOEVELTESNCOLOTLDCBO
AUVESIBABOEBIBLOOEVELTESNCOLOTLDCBO
\end{lstlisting}

\section*{Answer - 2:}

\textbf{Listing 3} contains the python code to perform encryption and decryption using the block cipher scheme mentioned in the question. 

\lstset{caption={ex1.2.py}}
\begin{lstlisting}
ROUNDS = 2


def key_function(K, i):
    return K + 75 * (i % 256)


def F(Ki, Pi):
    return 127 * Ki + (Pi % 256)


def encrypt(msg, key):
    Li = msg[0]
    Ri = msg[1]
    temp = 0
    for i in range(0, ROUNDS):
        Ki = key_function(key, i)
        temp = Li ^ F(Ki, Ri)
        Li = Ri
        Ri = temp
    return [Ri, Li]


def decrypt(cipher, key):
    Li = cipher[1]
    Ri = cipher[0]
    temp = 0
    for i in range(ROUNDS, 0, -1):
        Ki = key_function(key, (i - 1))
        temp = Ri ^ F(Ki, Li)
        Ri = Li
        Li = temp
    return [Li, Ri]


print(encrypt([86, 83], 89))
\end{lstlisting}
\textbf{Listing 4} contains the output of the python code in \textbf{Listing 3} containing the encrypted cipher-text "\textit{[20955, 11308]}".

\lstset{caption={output of ex1.2.py}}
\begin{lstlisting}
[20955, 11308]
\end{lstlisting}

\section*{Answer - 3:}

\textbf{Listing 5} contains the python code to perform encryption using a modified version of the DES implementation in Python.

The modified fork of the DES python library can be found at:\linebreak
\href{https://github.com/amar-laksh/pydes}{https://github.com/amar-laksh/pydes}

\lstset{caption={ex1.3.py}}
\begin{lstlisting}
import sys
sys.path.insert(1, 'pydes')
from pydes import des


def h2b(byte_string):
    return bytes.fromhex(byte_string)


key = h2b("0000000000000000")
msg = h2b("0000000000000000")

d = des(rounds=1)
cipher = d.encrypt(key, msg)

cipher = ' '.join(format(ord(x), 'b') for x in cipher)

print("The ciphertext after 1 round: %s" % cipher)
\end{lstlisting}
\textbf{Listing 6} contains the output of the python code in \textbf{Listing 5} containing the encrypted cipher-text "\textit{100 100 1 1010101 1010101 1 1010100 1010101}". 

\lstset{caption={output of ex1.3.py}}
\begin{lstlisting}
The ciphertext after 1 round: 100 100 1 1010101 1010101 1 1010100 1010101
\end{lstlisting}
This cipher-text is obtained even when the encryption key and plain-text is all zeroes.

\section*{Answer - 4:}

\textbf{Listing 7} contains the python code to perform encryption using a modified version of the DES implementation in Python.

The modified fork of the DES python library can be found at:\linebreak
\href{https://github.com/amar-laksh/pydes}{https://github.com/amar-laksh/pydes}

\lstset{caption={ex1.4.py}}
\begin{lstlisting}
import sys
sys.path.insert(1, 'pydes')
from pydes import des


def h2b(byte_string):
    return bytes.fromhex(byte_string)


key = h2b("0000000000000000")
x = h2b("0000000000000000")
y = h2b("0008000000000000")


for i in range(1, 17):
    print("ROUND %i:" % i)

    d = des(rounds=i)
    cipher_x = d.encrypt(key, x)
    cipher_y = d.encrypt(key, y)

    cipher_x = int.from_bytes(bytes(cipher_x, "UTF-8"), "little")
    cipher_y = int.from_bytes(bytes(cipher_y, "UTF-8"), "little")

    # XOR operation can be used to get the number of different bits
    diff_x_y = ((cipher_x) ^ (cipher_y))

    print(hex(cipher_x))
    print(hex(cipher_y))
    print("NUMBER OF BITS DIFFERENT: %s" % bin(diff_x_y).count('1'))

\end{lstlisting}
\textbf{Listing 8} contains the output of the python code in \textbf{Listing 7} containing the encrypted cipher-text and the number of bits different when two inputs are "x" and "y" respectively. 

\lstset{caption={output of ex1.4.py}}
\begin{lstlisting}
ROUND 1:
0x5554015555010404
0x5554115554010c04
NUMBER OF BITS DIFFERENT: 3
ROUND 2:
0xaec3adc356bac2bfc247594d
0xaec3adc377aac2b9c2420d49
NUMBER OF BITS DIFFERENT: 11
ROUND 3:
0x99c29ec3bdc2707e8bc2b7c38ec3
0x98c39ac2abc2112291c35f92c3
NUMBER OF BITS DIFFERENT: 57
ROUND 4:
0x26bdc32ea1c3a8c353afc399c2
0xa4c224477741b7c3aac2a4c3
NUMBER OF BITS DIFFERENT: 38
ROUND 5:
0x49afc20d96c294c3a2c28ec376
0x5c4d8ac3bbc386c3bac34589c2
NUMBER OF BITS DIFFERENT: 36
ROUND 6:
0x96c35b5a68bdc34089c2bcc3
0xb9c29fc390c2b6c39dc3b5c29bc252
NUMBER OF BITS DIFFERENT: 60
ROUND 7:
0xb8c2b3c2a0c291c3bfc391c216bcc3
0x77afc271bcc3afc26e63b5c3
NUMBER OF BITS DIFFERENT: 55
ROUND 8:
0x613311b7c3bac2223cacc2
0xbfc30eb2c2b8c31b89c286c3aec3
NUMBER OF BITS DIFFERENT: 53
ROUND 9:
0x82c32322bbc274042818
0xafc20821b4c327178dc29cc2
NUMBER OF BITS DIFFERENT: 53
ROUND 10:
0x90c2174137a9c2490074
0x1a0457b8c25a3b4e2c
NUMBER OF BITS DIFFERENT: 36
ROUND 11:
0x747b93c36f1393c251bcc3
0x600cbec330a4c23298c349
NUMBER OF BITS DIFFERENT: 45
ROUND 12:
0xbcc3b2c2b2c28fc32266b7c3bcc2
0x80c35cbcc2250c61b5c382c3
NUMBER OF BITS DIFFERENT: 48
ROUND 13:
0xacc271249ac35198c2afc33c
0x84c3bdc23c1f4d97c3bbc280c3
NUMBER OF BITS DIFFERENT: 59
ROUND 14:
0x1ca2c20ca0c2b7c2748bc26c
0x99c27f392b8ec2bbc32381c3
NUMBER OF BITS DIFFERENT: 50
ROUND 15:
0x791158407eacc2538cc3
0x72bfc326174ca6c34296c3
NUMBER OF BITS DIFFERENT: 34
ROUND 16:
0xa7c223b1c281c3a9c34da6c28cc2
0xa1c2abc35d6e8dc28cc390c3adc2
NUMBER OF BITS DIFFERENT: 46
\end{lstlisting}


\end{document}
