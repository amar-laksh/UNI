\documentclass[10pt,a4paper,oneside]{article}
\usepackage[utf8]{inputenc}
\usepackage{amsmath}
\usepackage{amsfonts}
\usepackage{amssymb}
\usepackage{makeidx}
\usepackage{graphicx}
\usepackage{lmodern}
\usepackage{listings}
\usepackage{color}


\lstset{language=Python}
\lstset{frame=lines}
\lstset{label={lst:code_direct}}
\lstset{basicstyle=\footnotesize}

\author{Amar Lakshya}
\title{Cryptography module, Exercises 1 (unassessed) - Answers}
\begin{document}

\part*{Cryptography module, Exercises 1 (unassessed) - Answers}
\section*{Answer - 1:}

\textbf{Listing 1} contains python code to brute-force the given cipher-text "\textit{AVUEVLETSEISBNACBOOLEOBTILBDLCOBOOE}".

\lstset{caption={ex1.1.py}}
\begin{lstlisting}
from math import ceil

def encode(msg, key):
    cipher = ""
    for rails in range(0, key):
        for char in range(rails, len(msg), key):
            cipher += msg[char]
    return cipher

def decode(cipher, key):
    msg = ""
    count = ceil(len(cipher)/key)
    for rails in range(0, count):
        for c in range(rails, len(cipher), count):
            msg += cipher[c]
    return msg

def crack_cipher(cipher):
    for key in range(1, len(cipher)):
        print(decode(cipher, key))

cipher = "AVUEVLETSEISBNACBOOLEOBTILBDLCOBOOE"

crack_cipher(cipher)
\end{lstlisting}
\textbf{Listing 2} contains the output of the python code in \textbf{Listing 1} containing the message, "\textit{ALICELOVESBOBBUTBOBDOESNOTLOVEALICE}" shown in the bold.


\lstset{caption={output of ex1.1.py}}
\lstset{keywordstyle=\bfseries}
\lstset{morekeywords={ALICELOVESBOBBUTBOBDOESNOTLOVEALICE}}
\begin{lstlisting}
AVUEVLETSEISBNACBOOLEOBTILBDLCOBOOE
AOVLUEEOVBLTEITLSBEDILSCBONBAOCOBEO
ABIVNLUABECDVBLLOCEOOTLBSEOEOOIBEST
AEODVILLUSECEBOOVNBBLATOECIOTBLESOB
ATAOLVSCBCUEBTOEIOIBVSOLOLBLBOENEDE
AEBOIOVTNLLBUSAEBOEECODOVIBBLELSOTC
ALICELOVESBOBBUTBOBDOESNOTLOVEALICE
ALICELOVESBOBBUTBOBDOESNOTLOVEALICE
AVSBBEILOVLENOOLCOUEIAOBBOEETSCLTDB
AVSBBEILOVLENOOLCOUEIAOBBOEETSCLTDB
AVSBBEILOVLENOOLCOUEIAOBBOEETSCLTDB
AEEEBCOOIDOOVVTINBLBLLBEULSSAOETBCO
AEEEBCOOIDOOVVTINBLBLLBEULSSAOETBCO
AEEEBCOOIDOOVVTINBLBLLBEULSSAOETBCO
AEEEBCOOIDOOVVTINBLBLLBEULSSAOETBCO
AEEEBCOOIDOOVVTINBLBLLBEULSSAOETBCO
AEEEBCOOIDOOVVTINBLBLLBEULSSAOETBCO
AUVESIBABOEBIBLOOEVELTESNCOLOTLDCBO
AUVESIBABOEBIBLOOEVELTESNCOLOTLDCBO
AUVESIBABOEBIBLOOEVELTESNCOLOTLDCBO
AUVESIBABOEBIBLOOEVELTESNCOLOTLDCBO
AUVESIBABOEBIBLOOEVELTESNCOLOTLDCBO
AUVESIBABOEBIBLOOEVELTESNCOLOTLDCBO
AUVESIBABOEBIBLOOEVELTESNCOLOTLDCBO
AUVESIBABOEBIBLOOEVELTESNCOLOTLDCBO
AUVESIBABOEBIBLOOEVELTESNCOLOTLDCBO
AUVESIBABOEBIBLOOEVELTESNCOLOTLDCBO
AUVESIBABOEBIBLOOEVELTESNCOLOTLDCBO
AUVESIBABOEBIBLOOEVELTESNCOLOTLDCBO
AUVESIBABOEBIBLOOEVELTESNCOLOTLDCBO
AUVESIBABOEBIBLOOEVELTESNCOLOTLDCBO
AUVESIBABOEBIBLOOEVELTESNCOLOTLDCBO
AUVESIBABOEBIBLOOEVELTESNCOLOTLDCBO
AUVESIBABOEBIBLOOEVELTESNCOLOTLDCBO
\end{lstlisting}

\section*{Answer - 2:}

\textbf{Listing 3} contains the python code to perform encryption and decryption using the block cipher scheme mentioned in the question. 

\lstset{caption={ex1.2.py}}
\begin{lstlisting}
ROUNDS = 2


def key_function(K, i):
    return K + 75 * (i % 256)


def F(Ki, Pi):
    return 127 * Ki + (Pi % 256)


def encrypt(msg, key):
    Li = msg[0]
    Ri = msg[1]
    temp = 0
    for i in range(0, ROUNDS):
        Ki = key_function(key, i)
        temp = Li ^ F(Ki, Ri)
        Li = Ri
        Ri = temp
    return [Ri, Li]


def decrypt(cipher, key):
    Li = cipher[1]
    Ri = cipher[0]
    temp = 0
    for i in range(ROUNDS, 0, -1):
        Ki = key_function(key, (i - 1))
        temp = Ri ^ F(Ki, Li)
        Ri = Li
        Li = temp
    return [Li, Ri]


print(encrypt([86, 83], 89))
\end{lstlisting}
\textbf{Listing 4} contains the output of the python code in \textbf{Listing 3} containing the encrypted cipher-text.

\lstset{caption={output of ex1.2.py}}
\begin{lstlisting}
[20955, 11308]
\end{lstlisting}

\end{document}
